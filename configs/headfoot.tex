%---------------------------------------------------------------------------------------
%
%	Configuração dos cabeçalho e rodapé
%
%---------------------------------------------------------------------------------------


%---------------------------------------------------------------------------------------
%	Carrega configuração básica
%---------------------------------------------------------------------------------------
\pagestyle{fancy}

%---------------------------------------------------------------------------------------
%	Estilo para o capítulo nos 
%---------------------------------------------------------------------------------------
\renewcommand{\chaptermark}[1]{
	\markboth{\color{NordPolarNight1}\normalsize\chaptername\ \thechapter.\ #1}{}
}

%---------------------------------------------------------------------------------------
%	Estilo para o capítulo corrente
%---------------------------------------------------------------------------------------
\renewcommand{\sectionmark}[1]{
	\markright{\color{NordPolarNight1}\normalsize Seção \thesection\hspace{5pt}#1}{}
} 

%---------------------------------------------------------------------------------------
%	Limpa as configurações padrão do Fancy
%---------------------------------------------------------------------------------------
\fancyhf{} 

%---------------------------------------------------------------------------------------
%	Insere o número da página esquerda e na direita
%---------------------------------------------------------------------------------------
%\fancyhead[LE,RO]{\normalsize\thepage}

%---------------------------------------------------------------------------------------
%	Insere o capítulo atual na direita das páginas pares
%---------------------------------------------------------------------------------------
\fancyhead[LO]{\rightmark}

%---------------------------------------------------------------------------------------
%	Insere a seção atual na esquerda das páginas pares
%---------------------------------------------------------------------------------------
\fancyhead[RE]{\leftmark}

%---------------------------------------------------------------------------------------
%	Insere o número da página de forma centralizada no rodapé das páginas
%---------------------------------------------------------------------------------------
\fancyfoot[C]{\color{NordPolarNight1}\normalsize\thepage}

%---------------------------------------------------------------------------------------
%	Define a largura da linha no cabeçalho das páginas
%---------------------------------------------------------------------------------------
\renewcommand{\headrulewidth}{2pt} 

%---------------------------------------------------------------------------------------
%	Configurando a impressão do título e número no início de cada capítulo
%---------------------------------------------------------------------------------------
\newlength\ChapWd
\settowidth\ChapWd{\huge\chaptertitlename}

\titleformat{\chapter}[display]
{\normalfont\filcenter\sffamily}
{\tikz[remember picture,overlay]
	{
		\node[fill=NordFrost4,font=\fontsize{60}{75}\selectfont\color{white},anchor=north east,minimum size=\ChapWd] 
		at ([xshift=-15pt,yshift=-45pt]current page.north east) 
		(numb) {\thechapter};
		\node[rotate=90,anchor=south,inner sep=0pt,font=\huge] at (numb.west) {\chaptertitlename};
	}
}{0pt}{\fontsize{33}{40}\selectfont\color{NordFrost4}#1}[\rule{\textwidth}{1.5pt}]
\titlespacing*{\chapter}
{0pt}{50pt}{10pt}

\makeatletter
\xpatchcmd{\ttl@printlist}{\endgroup}{{\noindent\color{NordFrost2}\rule{\textwidth}{1.5pt}}\vskip30pt\endgroup}{}{}
\makeatother


%---------------------------------------------------------------------------------------
%	Define o estilo plain para as páginas especiais do manuscrito
%---------------------------------------------------------------------------------------
\fancypagestyle{plain}{
	\fancyhead{}\renewcommand{\headrulewidth}{0pt}%
}

%---------------------------------------------------------------------------------------
%	Reconfigura o comando cleardoublepage
%---------------------------------------------------------------------------------------
\makeatletter
\renewcommand{\cleardoublepage}{
	\clearpage\ifodd\c@page\else
	\hbox{}
	\vspace*{\fill}
	\thispagestyle{empty}
	\newpage
	\fi
}