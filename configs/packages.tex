%---------------------------------------------------------------------------------------
% Arquivo para gerenciamento dos pacotes do projeto: vBook5
% Autor: Valdigleis (valdigleis@gmail.com)
%---------------------------------------------------------------------------------------

%---------------------------------------------------------------------------------------
% Pacote para padronização dos caracteres
%---------------------------------------------------------------------------------------
\usepackage[utf8]{inputenc}

%---------------------------------------------------------------------------------------
% Pacote para codificação final dos caracteres no documento de saída
%---------------------------------------------------------------------------------------
\usepackage[T1]{fontenc}

%---------------------------------------------------------------------------------------
% Pacote para idiomas (definindo padrão sendo o pt-br) e cuidar do hifém
%---------------------------------------------------------------------------------------
\usepackage[brazil]{babel}
\usepackage{csquotes}

%---------------------------------------------------------------------------------------
% Pacote para forçar ambiente flutuantes com H
%---------------------------------------------------------------------------------------
\usepackage{float}

%---------------------------------------------------------------------------------------
% Pacote para usar os logos do TeX
%---------------------------------------------------------------------------------------
\usepackage{metalogo}

%---------------------------------------------------------------------------------------
%	O pacote de icones Awesome
%---------------------------------------------------------------------------------------
\usepackage{fontawesome5}

%---------------------------------------------------------------------------------------
%	O pacote para o índice
%---------------------------------------------------------------------------------------
\usepackage{titletoc} 

%---------------------------------------------------------------------------------------
%	Os pacotes para estlizar os título das seções
%---------------------------------------------------------------------------------------
\usepackage[explicit]{titlesec}
\usepackage{xpatch}

%---------------------------------------------------------------------------------------
%	O pacote para configurar os títulos e os rodapés
%---------------------------------------------------------------------------------------
\usepackage{fancyhdr}

%---------------------------------------------------------------------------------------
%	O pacote para customizar as listas
%---------------------------------------------------------------------------------------
\usepackage{enumitem}

%---------------------------------------------------------------------------------------
% Pacote para melhorar o espaçamento e trata "overfull" (ou "underfull") boxes
%---------------------------------------------------------------------------------------
\usepackage{microtype}

%---------------------------------------------------------------------------------------
% Pacote para definir cores
%---------------------------------------------------------------------------------------
\usepackage[dvipsnames]{xcolor}

%---------------------------------------------------------------------------------------
% Pacote para definir cores nas linhas das tabelas
%---------------------------------------------------------------------------------------
\usepackage[dvipsnames]{colortbl}

%---------------------------------------------------------------------------------------
%	O pacote para identar a primeira linha dos capítulos e seções
%---------------------------------------------------------------------------------------
\usepackage{indentfirst}

%---------------------------------------------------------------------------------------
% Pacote para fazer correções de layout de documento
%---------------------------------------------------------------------------------------
%\usepackage{layouts}

%---------------------------------------------------------------------------------------
%	O pacote para setar espaçamento
%---------------------------------------------------------------------------------------
%\usepackage{setspace}

%---------------------------------------------------------------------------------------
%	O pacote para importar imagens como sendo o fundo do documento
%---------------------------------------------------------------------------------------
%\usepackage{eso-pic}
%\usepackage{wallpaper}

%---------------------------------------------------------------------------------------
%	O pacote para a geometria do documento
%---------------------------------------------------------------------------------------
\usepackage{geometry}

%---------------------------------------------------------------------------------------
%	O pacote para a inserir símbolos de parágrafo e seção ao texto
%---------------------------------------------------------------------------------------
\usepackage{textcomp}

%---------------------------------------------------------------------------------------
% Pacote para desenhar dados
%---------------------------------------------------------------------------------------
\usepackage{customdice}

%---------------------------------------------------------------------------------------
% Pacote para desenhos
%---------------------------------------------------------------------------------------
\usepackage{tikz}

%---------------------------------------------------------------------------------------
% Pacote para desenhos
%---------------------------------------------------------------------------------------
\usepackage{tikz-qtree}

%---------------------------------------------------------------------------------------
%	O pacote de desenho circuitikz
%---------------------------------------------------------------------------------------
\usepackage[RPvoltages]{circuitikz}

%---------------------------------------------------------------------------------------
%	Ativar bibliotecas específicas
%---------------------------------------------------------------------------------------
\usetikzlibrary{positioning, calc, chains, fit, shapes, automata, trees, decorations.pathreplacing,backgrounds}

%---------------------------------------------------------------------------------------
%	O pacote para desenhar maps de Karnaugh
%---------------------------------------------------------------------------------------
\usepackage{karnaugh-map}

%---------------------------------------------------------------------------------------
% Pacote para referência cruzada e link de referência
%---------------------------------------------------------------------------------------
\usepackage{hyperref} 

%---------------------------------------------------------------------------------------
%	O pacote para epígrafe
%---------------------------------------------------------------------------------------
\usepackage{epigraph}

%---------------------------------------------------------------------------------------
%	O pacote de plotar gráficos 
%---------------------------------------------------------------------------------------
\usepackage{pgfplots}

%---------------------------------------------------------------------------------------
%	O pacote para importar imagens
%---------------------------------------------------------------------------------------
\usepackage{graphicx} 

%---------------------------------------------------------------------------------------
%	O pacote para subcaptions
%---------------------------------------------------------------------------------------
\usepackage{subcaption}

%---------------------------------------------------------------------------------------
%	O pacote para pegar o tempo atual
%---------------------------------------------------------------------------------------
\usepackage[yyyymmdd,hhmmss]{datetime}

%---------------------------------------------------------------------------------------
%	O pacote para realizar cálculos
%---------------------------------------------------------------------------------------
\usepackage{calc}

%---------------------------------------------------------------------------------------
%	O pacote para multicoluna e multilinha
%---------------------------------------------------------------------------------------
\usepackage{multicol}
\usepackage{multirow}

%---------------------------------------------------------------------------------------
%	O pacote para adicionar a plaquinha
%---------------------------------------------------------------------------------------
\usepackage{manfnt}

%---------------------------------------------------------------------------------------
%	O pacote para escrever pseudo códigos
%---------------------------------------------------------------------------------------
\usepackage[portuguese,ruled,lined, linesnumbered]{algorithm2e}
\usepackage{algorithmic}

%---------------------------------------------------------------------------------------
%	O pacote para escrever e importar códigos de linguagens de programação
%---------------------------------------------------------------------------------------
\usepackage{listings}

%---------------------------------------------------------------------------------------
%	O pacote para múltiplas referências bibliográficas
%---------------------------------------------------------------------------------------
\usepackage{chapterbib}

%---------------------------------------------------------------------------------------
%	O pacote e configuração para a referência
%---------------------------------------------------------------------------------------
\usepackage{natbib}

%---------------------------------------------------------------------------------------
%	Os pacotes da matemática
%---------------------------------------------------------------------------------------
\usepackage{amsmath}
\usepackage{amsfonts}
\usepackage{amssymb}
\usepackage{amsthm}
\usepackage{mathtools}
\usepackage{wasysym}

%---------------------------------------------------------------------------------------
%	O pacote para construção dos blocos de provas
%---------------------------------------------------------------------------------------
\usepackage{logicproof}

%---------------------------------------------------------------------------------------
%	O pacote para deduçao natural usando diagramas de Fitch
%---------------------------------------------------------------------------------------
\usepackage{fitch}

%---------------------------------------------------------------------------------------
%	O pacote para construção árvores de derivação
%---------------------------------------------------------------------------------------
\usepackage{proof}

%---------------------------------------------------------------------------------------
%	O pacote para adicionar o not antes das relações
%---------------------------------------------------------------------------------------
\usepackage{centernot}

%---------------------------------------------------------------------------------------
%	O pacote para criar ambientes de caixa usando \newmdenv
%---------------------------------------------------------------------------------------
\RequirePackage[framemethod=default]{mdframed}

%---------------------------------------------------------------------------------------
%	O pacote para corte diagonal nas células de uma matriz
%---------------------------------------------------------------------------------------
\usepackage{diagbox}
