%----------------------------------------------------------------------------------------
%	Página de Disclaimer
%----------------------------------------------------------------------------------------
\begingroup

\newpage \thispagestyle{empty} \
\newpage

\thispagestyle{empty}
\begin{center}
	{\normalfont\fontsize{20}{20}\sffamily\selectfont {\color{NordAurora5}\faLightbulb} \\ \textbf{\textit{Sobre este documento}}}\par
\end{center}

\vspace{0.75cm}


Há muito tempo, em uma galáxia muito, muito distante, quando eu era aluno de graduação, as disciplinas ligadas à grande área chamada de matemática discreta eram, em geral, lecionadas com um enforque totalmente formal e analógico\footnote{Tudo era feito com lápis, papel e borracha.}. Para mim, isso era maravilhoso, pois amo matemática, entretanto, o tempo passou, o mundo mudou e hoje sou o professor e não o aluno. Hoje, a mentalidade e os interesses dos alunos são diferentes dos que eu tinha, há uma necessidade de ver as coisas visualmente, na prática, o mundo abstrato da matemática não tem mais tanto charme quando tinha na minha época, por esse motivo, decide escrever esse material.

Meu objetivo para este material, é que ele seja, uma fonte de consulta para aprender os tópicos da matemática discreta de forma interativa, usando a linguagem Haskell como ferramenta para implementações práticas dos tópicos. Entre esses tópicos estarão:

%Este documento vem sendo construído aos poucos ({\color{NordAurora1}em passos de tartaruga}), tendo como base diversas notas de aula (manuscritas a mão) que eu preparei  para ministrar cursos de graduação nos seguintes tópicos:


\begin{multicols}{2}
	\begin{fieldsList}
		\item Conjuntos, relações e funções;
		\item Introdução à Lógica;
%		\item Álgebra universal;
		\item Teoria dos códigos;
		\item Linguagem formais e autômatos;
		\item Computabilidade e decidibilidade;
%		\item Análise de algoritmos;
		\item Teoria de Grafos;
    \item Tipos abstrados de dados; 
		\item Fundamentos de Categorias;
%		\item Teoria da Informação;
	\end{fieldsList}
\end{multicols}	

Espero que esse material seja útil para você caro leitor, e se possível, gostaria de solicitar sua ajuda para melhora cada vez mais este material! Pois, uma vez que, este material ainda é um projeto em andamento e possivelmente sua escrita nunca será realmente concluída com total aprovação minha, tenho certeza que você poderá encontrar diversos erros, que gentilmente gostaria de solicitar que você leitor me envie por e-mails\footnote{E-mail do autor: \url{valdigleis@gmail.com}} ou \textit{issues}\footnote{Páginas de \textit{issues}: \url{https://gitlab.com/valdigleis/mcf/-/issues}} reports de tais erros, no caso de ser meu aluno também pode fazer apontamentos através da comunidade \textbf{extra-classe}\footnote{Acessível através do link \url{https://valdigleis.site/extraclasse}}. Também, agradeço por qualquer sugestão para melhorar o texto.

%\begin{figure}[H]
%	\centering
%	\includegraphics[scale=0.3]{fig/Alicia-C.png}
%	\caption{ALiCIA chamando sua atenção para o parágrafo abaixo.}
%\end{figure}

%Para finalizar, a personagem que você irá encontrar de forma recorrente neste documento chama-se ALiCIA\footnote{ALiCIA é um acrônimo para Autômatos, Linguagens, Complexidade, Informação e Algoritmos.}, ela é uma criação do autor deste livro e todas as imagens da mesma são de propriedade do autor, não sendo permitido o usado das imagens por terceiros sem autorização assinada pelo autor deste documento.

\endgroup
\newpage
